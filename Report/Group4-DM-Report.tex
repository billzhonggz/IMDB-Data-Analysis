\documentclass[10pt]{article}
    \usepackage{CJKutf8}
    \author{Group 4}
    \title{Analysis on IMDb Database}
    \begin{document}
    \begin{CJK}{UTF8}{gbsn}
        \maketitle
        \tableofcontents
        \section{Group Information}
        \begin{tabular} {|c|c|c|}
            \hline
            English Name & Chinese Name & Student ID \\
            \hline
            Jeff & 傅永升 & 1430003004 \\
            \hline
            Covey & 刘克盾 & 1430003011 \\
            \hline
            Garfield & 邬嘉祺 & 1430003029 \\
            \hline
            Frank & 邬可夫 & 1430003030 \\
            \hline
            Bill & 钟钧儒 & 1430003045 \\
            \hline
        \end{tabular}
        \section{Project Introduction}
        In this section, information of the project will be introduced.
            \subsection{Project Title}
            Analysis on IMDb Database.
            \subsection{Goals}
            By analyzing the dataset from the IMDb, find some associations between columns, classcify movies according to some criterias.
        \section{Data \& Data Collections}
        In this section, information of the dataset used in this project will be introduced.
            \subsection{Dataset}
            Here are some information of the dataset itself.
            \begin{itemize}
                \item Formate: csv file.
                \item Size: 302KB
                \item Number of rows: 1000
                \item Number of columns: 12
                \begin{itemize}
                    \item Rank
                    \item Title
                    \item Genre
                    \item Description
                    \item Director
                    \item Actors
                    \item Year
                    \item Runtime
                    \item Rating
                    \item Votes
                    \item Revenue
                    \item Metascore
                \end{itemize}
                \item Source: IMDb (https://www.imdb.com)
            \end{itemize}
            \subsection{Data Collection}
            \paragraph{Internet Spider}
            The most common way to collect such dataset is to use Internet spiders. In this case, a Python packge called Scarpy was used. 
        \section{Analysis}
        Raised by members in the group, attributes are analyzed by members individually.
            \subsection{Analysis on Directors}
            The section is provided by Junru (Bill) Zhong.
            This analysis aims to find out the following associations.
            \begin{itemize}
                \item Find out which director earned the largest revenue.
                \item Find out the gerene of films of the first three directors with the largest revenue.
                \item Estimate how much they will earn for their next film.
            \end{itemize}
                \subsubsection{Code implementation}
                In this section, Python (Anaconda) was used, with package Pandas.
                \\
                The steps are listed below.
                \begin{enumerate}
                    \item List all directors.
                    \item Get the total revenues each directors got.
                    \item Rank all directors by revenues got each film in average.
                    \item List information of directed film(s) of the top-3 directors.
                    \item Estimate the revenues of their next film by some algorithms.
                \end{enumerate}
            \subsection{Revenue and Genre}
            This section is provided by Yongsheng (Jeff) Fu.
            This analysis is to find out the change in taste of genre from 2006 to 2016.
    \end{CJK}
    \end{document}